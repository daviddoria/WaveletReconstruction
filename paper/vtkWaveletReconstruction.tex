%
% Complete documentation on the extended LaTeX markup used for Insight
% documentation is available in ``Documenting Insight'', which is part
% of the standard documentation for Insight.  It may be found online
% at:
%
%     http://www.itk.org/

\documentclass{InsightArticle}

\usepackage[dvips]{graphicx}
\usepackage{float}
\usepackage{filecontents}

\begin{filecontents}{shortbib.bib}
@ARTICLE{reconstruction,
    AUTHOR={J. Manson \and G. Petrova \and S. Schaefer},
    TITLE={Streaming Surface Reconstruction Using Wavelets},
    JOURNAL={Eurographics Symposium on Geometry Processing},
    YEAR={2008},
}
\end{filecontents}

\usepackage{subfigure}

%%%%%%%%%%%%%%%%%%%%%%%%%%%%%%%%%%%%%%%%%%%%%%%%%%%%%%%%%%%%%%%%%%
%
%  hyperref should be the last package to be loaded.
%
%%%%%%%%%%%%%%%%%%%%%%%%%%%%%%%%%%%%%%%%%%%%%%%%%%%%%%%%%%%%%%%%%%
\usepackage[dvips,
bookmarks,
bookmarksopen,
backref,
colorlinks,linkcolor={blue},citecolor={blue},urlcolor={blue},
]{hyperref}


\title{Wavelet Surface Reconstruction for VTK}

% 
% NOTE: This is the last number of the "handle" URL that 
% The Insight Journal assigns to your paper as part of the
% submission process. Please replace the number "1338" with
% the actual handle number that you get assigned.
%
\newcommand{\IJhandlerIDnumber}{3149}

% Increment the release number whenever significant changes are made.
% The author and/or editor can define 'significant' however they like.
\release{0.00}

% At minimum, give your name and an email address.  You can include a
% snail-mail address if you like.

\author{David Doria and Arnaud Gelas}
%\author{David Doria}
%\authoraddress{Rensselaer Polytechnic Institute, Troy NY \and Harvard University, Cambridge MA}


\begin{document}

%
% Add hyperlink to the web location and license of the paper.
% The argument of this command is the handler identifier given
% by the Insight Journal to this paper.
% 
\IJhandlefooter{\IJhandlerIDnumber}


\ifpdf
\else
   %
   % Commands for including Graphics when using latex
   % 
   \DeclareGraphicsExtensions{.eps,.jpg,.gif,.tiff,.bmp,.png}
   \DeclareGraphicsRule{.jpg}{eps}{.jpg.bb}{`convert #1 eps:-}
   \DeclareGraphicsRule{.gif}{eps}{.gif.bb}{`convert #1 eps:-}
   \DeclareGraphicsRule{.tiff}{eps}{.tiff.bb}{`convert #1 eps:-}
   \DeclareGraphicsRule{.bmp}{eps}{.bmp.bb}{`convert #1 eps:-}
   \DeclareGraphicsRule{.png}{eps}{.png.bb}{`convert #1 eps:-}
\fi


\maketitle


\ifhtml
\chapter*{Front Matter\label{front}}
\fi


% The abstract should be a paragraph or two long, and describe the
% scope of the document.
\begin{abstract}
\noindent
This document presents an implementation of the Wavelet surface reconstruction algorithm in the VTK framework. (This code was adapted directly from the original implementation by Josiah Manson \cite{reconstruction}, with permission). We present a class, $vtkWaveletReconstruction$, which produces a surface from an oriented point set. A Paraview plugin interface is provided to allow extremely easy experimentation with the new functionality. We propose these classes as an addition to the Visualization Toolkit.

\end{abstract}

\IJhandlenote{\IJhandlerIDnumber}

\tableofcontents

\section{Introduction}
There are several data acquisition methods which produce 3D points as output. Examples include Light Detection and Ranging (LiDAR) scanners, Structure From Motion (SFM) algorithms, and Multi View Stereo (MVS) algorithms. Producing a mesh from these points if often of interest. This paper presents an algorithm for producing a surface from a set of 3D points which have an associated normal vector. This method was originally published in \cite{reconstruction}, we have simply implemented it in the VTK framework.

\section{vtkWaveletReconstruction}

\subsection{Options}



%%%%%%%%%%%%%%%%%%%%%%%%%%%%%%%%%%
\subsection{Demonstration}
To demonstrate surface reconstruction, we use points sampled from a sphere. We computed the normals of these points using a previously submitted class, vtkPointSetNormalEstimation. We oriented these normals using another previously submitted class, vtkPointSetNormalOrientation. The resulting oriented point set is shown in Figure \ref{fig:SpherePoints}. The signed distance volume, an intermediate step, is shown for clarity in \ref{fig:SignedDistances}. In Figure \ref{fig:SphereSurface}, we show the final reconstructed surface.
% 
% \begin{figure}[ht!]
% \centering
% \subfigure[Oriented point set]{
% \includegraphics[width=0.3\linewidth]{Images/SpherePoints}
% \label{fig:SpherePoints}
% }
% \subfigure[Interpolated volume of signed distances]{
% \includegraphics[width=0.3\linewidth]{Images/SignedDistances}
% \label{fig:SignedDistances}
% }
% \subfigure[Sphere surface]{
% \includegraphics[width=0.3\linewidth]{Images/SphereSurface}
% \label{fig:SphereSurface}
% }
% \caption{Surface reconstruction demonstration.}
% \label{fig:SurfaceReconstruction}
% \end{figure}

\subsection{Code Snippet}
\begin{verbatim}

\end{verbatim}

%%%%%%%%%%%%%%%%%%%%%
\section{Future Work}
The algorithm implemented in this paper is very sensitive to noise error in the normal computation, the grid spacing, and the density of the points. There are several newer techniques, such as Poisson surface reconstruction, that we intend to implement for VTK in the future.

%%%%%%%%%%%%%%%%%%%%%
\pagebreak
\section{Paraview Plugin}
For convenience, this code is shipped with a Paraview filter plugin. The plugin provides an easy way to set the parameters as well as integrate the new code into your workflow. A screenshot of the plugin interface is shown in Figure \ref{fig:Plugin}.

% \begin{figure}[H]
% \centering
% \includegraphics[width=0.3\linewidth]{Images/Plugin}
% \label{fig:Plugin}
% \caption{Paraview plugin screenshot}
% \end{figure}

\bibliographystyle{plain}
\bibliography{shortbib}


\end{document}